Analizando el estado del arte de la aplicación, se encuentran herramientas que permiten analizar y explotar binarios de forma automática, sin embargo, no se ha conseguido localizar programas que permitan la generación automática de los retos.

La ausencia de herramientas específicas para la generación automatizada de retos destaca una oportunidad significativa en el campo de la ciberseguridad educativa. El desarrollo de una herramienta que automatice este proceso no solo facilitaría la enseñanza de la explotación de binarios, sino que también permitiría a los estudiantes enfrentarse a una amplia gama de vulnerabilidades en un entorno seguro y controlado. Esta innovación puede ayudar a llenar el vacío existente en los recursos educativos disponibles, proporcionando un enfoque más práctico y enfocado en la formación de futuros profesionales de ciberseguridad.