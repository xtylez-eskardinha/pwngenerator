En el análisis del estado del arte de las herramientas de ciberseguridad, se identifican diversas aplicaciones que permiten analizar y explotar binarios de forma automática.
Entre las más relevantes se encuentran `pwntools' la cual es un framework de explotación que permite automatizar muchas funcionalidades y tiene una función de autoexplotación de binarios mediante `Corefiles'.
Otra herramienta es `rex' del equipo de `angr', sin embargo no es sencilla de instalar además de que carece de estabilidad en su ejecución.
Estas herramientas son eficaces para la identificación de vulnerabilidades y la explotación automatizada, pero se enfocan principalmente en la evaluación de binarios existentes y no en la creación de desafíos personalizados para la enseñanza.

Sin embargo, no se han encontrado programas que ofrezcan la generación automática de retos de explotación binaria a partir de códigos C legítimos.
Esta carencia subraya una oportunidad significativa en el campo de la ciberseguridad educativa, donde hay una necesidad creciente de recursos didácticos que permitan a los estudiantes practicar de manera segura con escenarios realistas.
El desarrollo de una herramienta que automatice la generación de estos retos no solo podría revolucionar la enseñanza de la explotación de binarios, sino que también facilitaría un aprendizaje práctico, permitiendo a los estudiantes enfrentarse a una variedad de vulnerabilidades en un entorno seguro y controlado.

Este proyecto responde a esa necesidad al proporcionar un sistema que convierte códigos C seguros en desafíos de explotación.
Al hacerlo, se ofrece una solución innovadora que fomenta un enfoque más práctico y orientado a la formación en materia de ciberseguridad, brindándoles la experiencia y habilidades necesarias para identificar y mitigar vulnerabilidades en binarios.