Con el paso del tiempo, la ciberseguridad se ha convertido en un campo de gran relevancia en el sector de las telecomunicaciones e informática.
Desde la invención y adopción de internet, los equipos informáticos están cada día más expuestos al mundo exterior.
Por ello, es de vital importancia asegurarse de que estos dispositivos sean seguros y no tengan vulnerabilidades que puedan permitir a un actor malintencionado tomar el control.

La vulnerabilidad del tipo `\acrfull{bof}' o `Desbordamiento de buffer', fue una de las primeras utilizadas para explotar sistemas remotos sin la interacción de los usuarios.
Este mecanismo de explotación fue documentado por primera vez en 1972 por la \acrfull{usaf}.
Sin embargo, no fue hasta 1988 cuando se descubrió el primer gusano llamado Morris \cite{Morris}, el cual, entre las diferentes técnicas de explotación hacía uso de un \acrshort{bof} sobre el servicio `fingerd' \cite{Fingerd}.

Con el tiempo y debido a la aparición nuevos lenguajes de programación, los \acrshort{bof} se han vuelto menos comunes.
Esto se debe a que estos nuevos lenguajes abstraen al usuario de la gestión de la memoria e interactúan con el sistema de una forma más segura y controlada.
Sin embargo, esta vulnerabilidad sigue siendo crítica, ya que, aunque el lenguaje abstraiga al programador de la gestión de memoria, el propio compilador del lenguaje o alguna librería externa insertada en el código podría tener este tipo de errores.

Este Trabajo Fin de Estudios se centra en el desarrollo de un sistema automatizado para la generación de códigos en lenguaje C que contienen vulnerabilidades intencionales. Este sistema permite a los usuarios practicar diversas técnicas de explotación de binarios, como el \acrshort{bof}, la ejecución de código arbitrario y la manipulación de la memoria, en un entorno seguro y educativo.
Al proporcionar códigos vulnerables de manera automatizada, el proyecto busca facilitar el proceso de aprendizaje, permitiendo a los usuarios del programa enfocarse en la identificación y explotación de vulnerabilidades, en lugar de la creación manual de escenarios.

Para el proyecto se ha optado por utilizar software de código abierto.
Esta elección permite que cualquier persona interesada tenga acceso al código, pudiendo utilizarlo y modificarlo según sus necesidades.
Además, el programa se publica en 'GitHub', referenciado en el apéndice \ref{apendice:git}, con el propósito de fomentar un desarrollo colaborativo continuo.
Este enfoque busca mejorar las limitaciones actuales y añadir soporte para flujos más complejos, así como para comunicaciones a través de 'sockets'.